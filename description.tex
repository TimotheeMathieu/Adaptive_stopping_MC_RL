\documentclass{article}
\usepackage{amsmath, amsfonts, amssymb, amsthm, stmaryrd}
\usepackage{enumitem}
\usepackage{hyperref}
\usepackage{bbm}

\usepackage[verbose=true,letterpaper]{geometry}
\newgeometry{
  textheight=9.5in,
  textwidth=6in,
  top=0.5in,
  headheight=12pt,
  headsep=25pt,
  footskip=30pt
}

%%%%%% Commands and theorems
\theoremstyle{plain}
\newtheorem{Theorem}{Theorem}
\newtheorem{Proposition}{Proposition}
\newtheorem{Corollary}{Corollary}
\newtheorem{Lemma}{Lemma}

\theoremstyle{remark}
\newtheorem{Definition}{Definition}
\newtheorem{Assumptions}{Assumptions}

\renewcommand{\P}{\mathbb{P}}
\newcommand{\E}{\mathbb{E}}
\newcommand{\R}{\mathbb{R}}
\newcommand{\N}{\mathbb{N}}

\newcommand{\sign}{\text{sign}}
\newcommand{\1}{\mathbbm{1}}

\newcommand{\argmin}{\arg\min}

% To include the section number in the equation numbering:
\numberwithin{equation}{section}
\title{Adaptive stopping in Monte-Carlo evaluation for Deep RL}
\date{}
\begin{document}
\maketitle
\section{Description of the problem}
In Reinforcement Learning, we often use Monte-Carlo methods to evaluate the performances of an algorithm. In particular, if we denote $e(A)$ some evaluation of algorithm $A$ (for example it can be the mean of the cumulative reward over 100 episodes).
\section{Adaptive stopping using Group Sequential Testing}
\section{Algorithmic considerations}
\section{Experimental results}


\bibliographystyle{plain}
\bibliography{biblio_file}
\end{document}
